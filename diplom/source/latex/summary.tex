\chapter*{ЗАКЛЮЧЕННЯ}
\addcontentsline{toc}{chapter}{ЗАКЛЮЧЕННЯ}
\label{5decision::doc}\label{5decision:id1}

Підводячи загальний підсумок роботи необхідно зазначити.

По-перше, незважаючи на стрімкий розвиток інформаційних технологій, зокрема мов програмування, інструментів та засобів реалізації програмних продуктів використання низькорівневих мов програмування з відкритим управлінням памяттю складає значну долю в порівнянні з високорівневими мовами. Як відомо, саме ці мови і мають недоліки, які полягають у можливості модифікації поведінки зі сторони.

По-друге, на сьогодні існує низка інструментів аналізу вихідних текстів програм, яка дозволяє отримати багато корисної інформації про вразливі місця в коді, але ці засоби, в основному, являються не повними, або частіше за все, не дають повної картини, яка б дозволила в короткий термін прийняти рішення щодо спрямування роботи над тим чи іншим модулем.

По-третє, на основі проведених експертних сучасних метрик програмного забезпечення  перевага була віддана метрикам Холстеда, Маккейб та Джилба, котрий складає строгого математичний алгоритм, а також запропоновано використання змішаної оцінки виходячи з оцінок цих метрик та кількісті потенційно-небезпечних вразливостей вихідних тектстів програм для підвищення точності при організації кібернетичного впливу. Дана модель надає можливість на основі екстраполяції попередніх результатів давати оцінку успішності проведення кібернетичного впливу з використанням тієї чи іншої вразливостей.

По-четверте, розглянувши сучасні тенденції кібернетичного впливу, та зробивши аналіз поширених вразливостей програмного забезпечення, способи їх використання та засоби виявлення, можна зробити висновок що дана предметна область поки що мало досліджена та потребує більш ґрунтовного вивчення.

Запропонована модель забезпечує в короткий термін проведення оцінки коду, при цьому результати оцінки можуть бути зкореговані спеціалістом данної предметної області в процесі роботи, що дозволить в подальшому аналізі отримувати більш точні прогнози.

Запропоновано практичну реалізацію метою якої є забезпечення зручного та швидкого аналізу програмних продуктів для подальшої, ефективної роботи з ними.